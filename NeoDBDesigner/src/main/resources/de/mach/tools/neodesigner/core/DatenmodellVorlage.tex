\documentclass{scrartcl}
\usepackage[a4paper, total={190mm, 260mm}]{geometry}
\usepackage[utf8]{inputenc}
% Schoenere Schriftart laden
\usepackage[T1]{fontenc}
\usepackage{lmodern}
% Deutsche Silbentrennung verwenden
\usepackage[ngerman]{babel}
\usepackage[table]{xcolor}
\usepackage{multicol}
\usepackage{tabularx}
\usepackage{tikz}
\usepackage{imakeidx}
% Bessere Unterstuetzung fuer PDF-Features
\usepackage[breaklinks=true,hidelinks]{hyperref}
\usepackage{longtable}
% Abmessungen der Seite
\geometry{
  left=1cm,
  right=1cm,
  top=1cm,
  bottom=2.5cm,
}
% Überschriften enger 
\RedeclareSectionCommands[
  beforeskip=-.5\baselineskip,
  afterskip=.25\baselineskip
]{section,subsection}
% Inhaltsverzeichnis einrücken
\RedeclareSectionCommand[tocnumwidth=12mm]{subsection}
\newcommand\Tstrut{\rule{0pt}{0.5ex}}
\makeindex[title=Tabellenindex,intoc]
% dreispaltiger Index am Ende
\makeatletter
\renewenvironment{theindex}
  {\if@twocolumn
      \@restonecolfalse
   \else
      \@restonecoltrue
   \fi
   \setlength{\columnseprule}{0pt}
   \setlength{\columnsep}{35pt}
   \begin{multicols}{3}[\section{\indexname}]
   \markboth{\MakeUppercase\indexname}
            {\MakeUppercase\indexname}
   \thispagestyle{plain}
   \setlength{\parindent}{0pt}
   \setlength{\parskip}{0pt plus 0.3pt}
   \relax
   \let\item\@idxitem}
  {\end{multicols}\if@restonecol\onecolumn\else\clearpage\fi}
\makeatother

\title{$title}
\author{$author}
\date{\copyright\today}

\begin{document}
\maketitle
\begin{multicols}{2}
\tableofcontents
\end{multicols}
\clearpage
#foreach( $section in $datamodel.getSortedSections() )
% Für jede Sektion eine neue Seite erstellen
\newpage
#set( $newpage = false )
\section{$datamodel.getSectionOrPageTitle($section)}
#foreach( $page in $datamodel.getSortedPages($section) )
% eine neue Seite erstellen
#if( !$newpage )
#set( $newpage = true )
#else
\newpage
#end
\subsection{$datamodel.getSectionOrPageTitle($page)}
\begin{center}
\begin{multicols}{3}
\begingroup
\scriptsize
#foreach( $table in $datamodel.getSortedTables($page))
% Alle Tabellen einer Seite einfügen
\label{$table}
\index{$table}
\begin{tikzpicture}
\rowcolors{2}{white}{gray!25} 
\setlength\arrayrulewidth{1pt}
\node (table) [inner sep=0pt] {
\begin{tabularx}{59mm}{Xp{8mm}p{4mm}p{3mm}p{4mm}}
\multicolumn{5}{c}{\textbf{$table\Tstrut}} \\
\hline
#set( $strut = "\Tstrut" )
% Primary Key Felder am Anfang der Tabelle auflisten
#foreach( $column in $datamodel.getPrimaryKeyColumns($page, $table))
#if( $column.isNullable() )
#set( $nullable = "" )
#else
#set( $nullable = "REQ." )
#end
#if( $column.isForeignKey() )
#set( $isFK = "\scalebox{.5}[1.0]{\hyperref[$column.getForeignKeyTableName()]{(FK)}}" )
#else
#set( $isFK = "" )
#end
#if( $column.hasIndex() )
#set( $index = "\scalebox{.5}[1.0]{(" + $column.getIndexName() + ")}" )
#else
#set( $index = "" )
#end
\scalebox{.7}[1.0]{$column.getName()} & \scalebox{.7}[1.0]{$column.getDomain()} & \scalebox{.5}[1.0]{$nullable} & $isFK & $index $strut \\
#set( $strut = "" )				
#end
#if( !$datamodel.getNonPrimaryKeyColumns($page, $table).isEmpty())
\hline
#set( $strut = "\Tstrut" )
#end
% Alle nicht Primärfelder auflisten
#foreach( $column in $datamodel.getNonPrimaryKeyColumns($page, $table))
#if( $column.isNullable() )
#set( $nullable = "" )
#else
#set( $nullable = "REQ." )
#end
#if( $column.isForeignKey() )
#set( $isFK = "\scalebox{.5}[1.0]{\hyperref[$column.getForeignKeyTableName()]{(FK)}}" )
#else
#set( $isFK = "" )
#end
#if( $column.hasIndex() )
#set( $index = "\scalebox{.5}[1.0]{(" + $column.getIndexName() + ")}" )
#else
#set( $index = "" )
#end
\scalebox{.7}[1.0]{$column.getName()} & \scalebox{.7}[1.0]{$column.getDomain()} & \scalebox{.5}[1.0]{$nullable} & $isFK & $index $strut \\
#set( $strut = "" )
#end
\end{tabularx}
};
\draw [rounded corners=.5em] (table.north west) rectangle (table.south east);
\end{tikzpicture}
\vspace{2mm}
#end
\endgroup
\end{multicols}
\end{center}
#end
#end
\newpage
\section{Constraints}
\begin{small}
\begin{center}
 \rowcolors{2}{white}{gray!25}
 \footnotesize
 \begin{longtable}{ l  l  l  l  r }
  Constraint & Tabelle & Referenz Tabelle & Spalte & Ordnungsziffer \\
  \hline
#foreach( $fk in $datamodel.getFkList() )
#if($fk.getTableName().length()>0)
#set( $txt_tbl = "\hyperref[$fk.getTableName()]{$fk.getTableName()}" )
#set( $txt_reftbl = "\hyperref[$fk.getRefTableName()]{$fk.getRefTableName()}" )
#else
#set( $txt_tbl = "")    
#set( $txt_reftbl = "") 
#end
  $fk.getForeignkeyName() & $txt_tbl & $txt_reftbl & $fk.getColumnName() & $fk.getOrder() \\
#end
 \end{longtable}
\end{center}
\end{small}
\newpage
\printindex
\end{document}